\chapter{Related work}\label{related_work}

Ultra-wideband technology has become the focus of intensive research in the field of indoor localization due to its potential for high-precision ranging. However, despite its advantages, much of the current literature recognizes that UWB’s performance is fundamentally limited in NLoS scenarios~\cite{wang2025, Elikplim2024survey}. These conditions, while extremely common in indoor environments, introduce significant ranging errors that degrade localization accuracy, making NLoS mitigation one of the most active and critical areas of investigation. To address the problem of NLoS propagation, a variety of approaches have been developed, most notably those relying on NLoS identification and mitigation algorithms~\cite{Wang2023NLoS}. 

The identification algorithms have been broadly classified into three categories by many scholars: map-based, range-based, and channel-based~\cite{khodjaev2010survey}. The map-based solutions rely on the assumption that map data of the environment (e.g., geometries and attenuation factors of the obstacles) are known, and the propagation channel can be identified using these data~\cite{khodjaev2010survey}. For example, Hyun et al.~\cite{hyun2019uwb} propose the ray-tracing algorithm to obtain the range correction for NLoS-induced errors based on a given floor plan. The range-based methods typically rely on time series of range estimations to identify the channel state~\cite{khodjaev2010survey}. Horiba et al.~\cite{horiba2015improved} achieve this by comparing the averaged measurements and estimated distances, derived from the measurements' variance. The channel-based methods work by analyzing the channel information, primarily acquired from the Channel Impulse Response (CIR) data~\cite{khodjaev2010survey, marano2010nlos}. The remainder of this chapter will focus on the channel-based methods, as they are widely used for NLoS identification and mitigation purposes due to their ability to directly exploit the CIR data, which provides detailed information about the received signal and its propagation environment~\cite{Lee2023CIR, pei2024fcn}.

Early works in this field primarily focused on statistical methods, such as hypothesis testing and thresholding techniques, where signals are classified as either Line-of-Sight (LoS) or NLoS based on joint probability distributions of their strength or other measurable parameters~\cite{venkatesh2007non}, which are derived from the prior data~\cite{schroeder2007nlos}. However, these methods, despite their simplicity, often fail to accurately identify NLoS conditions, as it is challenging to precisely determine the probability distributions and unclear on how to set the threshold~\cite{shi2014rss}.

Over time, more advanced algorithms have emerged, such as those based on machine learning (ML) and deep learning (DL) approaches, which have gained popularity due to their ability to handle complex, non-linear relationships between features without a requirement for a priori knowledge~\cite{yu2018novel, pei2024fcn}. For instance, Musa et al.~\cite{musa2019decision} proposed a decision tree-based approach for NLoS detection from CIR features. Their method successfully demonstrated a high accuracy of over \SI{90}{\percent}. Ferreira et al.~\cite{ferreira2021feature} presented a CIR feature selection study and compared five different machine learning algorithms for channel state classification. While achieving promising results, their study is limited to static scenarios, lacking validation under dynamic conditions. Shoude Wang and Nur Syazreen Ahmad~\cite{wang2024robust} proposed a method based on Fuzzy Comprehensive Evaluation (FCE), which is employed for preliminary classification, and Extreme Gradient Boosting (XGBoost), used for refined classification. The results demonstrate a high average classification accuracy of \SI{92}{\percent}. However, the authors have not assessed the performance within location estimation models.

Traditional ML approaches often involve the use of handcrafted features, which may not accurately or comprehensively represent the signal characteristics~\cite{abbasi2021novel}. Therefore, many DL-based approaches have been proposed in recent years, due to their ability to automatically extract relevant features used in the decision-making process~\cite{shaheen2016impact}. The widely used approaches are commonly based on convolutional neural networks (CNN), long short-term memory (LSTM), or a combination of these architectures~\cite{pei2024fcn}. For instance, Stahlke et al.~\cite{stahlke2020nlos} propose different CNN-based approaches for channel state classification directly from the raw CIR data. The ResNet, Encoder, and fully-connected network architectures were evaluated, showing average accuracy of up to \SI{94}{\percent} with the ability to generalize to unseen environments. Jiang et al.~\cite{jiang2020uwb} proposed a CNN-LSTM stacked model for signal classification. Raw CIR data was directly fed into the CNN, acting like a feature extractor, and its outputs were consequently used as input for the LSTM classifier. However, the study focuses solely on channel state classification, without evaluating the impact of classification accuracy on the resulting localization performance.

Once NLoS measurement is identified, it must be processed accordingly in order to improve localization precision. Two common methods for mitigating NLoS effects include: (i) estimating the error in the ranging measurement and correcting for it, or (ii) employing specialized positioning techniques that account for NLoS conditions~\cite{yu2018novel}. The latter approach often employs Bayesian filters or algorithms like the weighted least squares (WLS) to improve signal-to-noise ratio and minimize the influence of NLoS-induced errors on the positioning system. Among commonly used filters are Kalman filters, particle filters, and their modifications~\cite{Wang2023NLoS}. Weighted approaches typically assign the NLoS measurements with lower weights, thereby limiting their contribution to location estimates~\cite{Wang2023NLoS}. These methods typically rely on NLoS classification or signal quality metrics to determine the confidence level of each measurement. Despite the advantage in simplicity offered by these methods, practical setups often do not provide a sufficient number of UWB links to permit the exclusion or down-weighting of some, as this can degrade positioning performance~\cite{ferreira2021feature}.

Sensor fusion techniques are commonly used to address this issue. Sensor fusion is used to combine UWB ranging data with complementary sensors that can provide additional information about the system's state by capturing motion dynamics, environmental features, or obstacles that are otherwise invisible to UWB measurements alone. Such sensors include Inertial Measurement Units (IMUs)~\cite{Kim2021IMU}, Light Detection and Ranging sensors (LiDARs)~\cite{Chen2022LIDAR}, cameras~\cite{Peng2022Visual}, or other technologies and their combinations. For example, the work by Ali et al.~\cite{Ali2021IMU} proposes a tightly coupled (TC) sensor fusion method, combining UWB ranging and IMU-based Pedestrian Dead Reckoning (PDR) using an Extended Kalman Filter (EKF), to improve indoor localization accuracy. Similarly, Chen et al.~\cite{Chen2022LIDAR} present a TC EKF-based UWB and LiDAR Simultaneous Localization and Mapping (LiDAR-SLAM) fusion model that uses LiDAR point clouds to filter out NLoS measurements. A recent work by Zhang et al.~\cite{zhang2023research} introduces a UWB/IMU fusion method using particle filtering with LoS/NLoS mapping and an occlusion error model, aimed at mitigating the NLoS errors caused by human occlusion. However, while sensor fusion methods provide substantial benefits, they often introduce challenges such as increased system cost, complexity, and computational overhead~\cite{Naheem2022IMU, wang2024comprehensive}.

Among the most recent advancements in NLoS mitigation are the models that aim to map the NLoS-induced errors to their features, offering a promising solution for error mitigation. This approach eliminates the necessity to exclude or downweight obstructed measurements and provides the opportunity to directly estimate error corrections. It is also known to be less environment-specific and to require less data to achieve comparable results~\cite{savic2015kernel, Simone2021UWB, Kim2023NLOS}. Simone et al.~\cite{Simone2021UWB} propose the REMNet, a deep neural network (DNN) model for error mitigation based on features of CIR data, improving the mean absolute error for NLoS signals by \SI{44.7}{\percent}. The work also addresses the widespread issue of high computation complexity and latency in ML-based approaches~\cite{Zeng2018NLOS} by introducing weight quantization and graph optimization. The authors also release a dataset for UWB ranging error mitigation in indoor environments, collected using Decawave DW1000 transceivers~\cite{simone_angarano_2020_4399187}. Tran et al.~\cite{Tran2022UWB} investigate the utility of various Two-way Ranging (TWR) UWB packets and their combinations as inputs to neural networks, presenting two novel data-driven approaches for error mitigation. Their methods demonstrate significant improvements over state-of-the-art models, reducing range errors by up to \SI{45}{\percent}. Additionally, they release a synchronized transaction-level CIR dataset that includes the CIR data of poll, response, and final TWR packets, which may aid future research. However, like in the majority of existing studies, the proposed dataset is based exclusively on Decawave DW1000 devices. While this ensures consistency, it limits hardware diversity and may hinder the generalization of proposed methods across different UWB platforms.

Some scholars have also adopted the approach of Transfer Learning (TL) to achieve better performance in unseen environments~\cite{fontaine2023transfer, li2023unsupervised}. For example, Fontaine et al.~\cite{fontaine2023transfer} use this concept by utilizing TL to adapt DNN models for error correction and NLoS detection across various environments, demonstrating up to \SI{50}{\percent} error reduction. However, practically all existing approaches still rely on labeled data to some extent, underscoring the requirement for datasets with labeled samples, which can be a logistical challenge in real-world applications~\cite{coppens2024removing}. A notable advancement in this area is the work by Yang et al.~\cite{yang2023self}, which proposes a self-supervised DNN for ranging error correction, significantly reducing the dependency on ground truth data. Building on the work of Fontaine et al.~\cite{fontaine2023transfer}, Coppens et al.~\cite{coppens2024removing} further advance the state-of-the-art, introducing a unique reward-based framework that uses CIR as a reinforcement learning agent state to iteratively refine range corrections based on the agreement between corrected ranges and EKF estimations. While demonstrating performance comparable to supervised methods, its effectiveness remains uncertain due to the reliance on predictable trajectories.

In conclusion, while UWB-based positioning systems offer significant advantages in terms of accuracy and robustness in multipath-prone environments, they remain susceptible to errors under NLoS conditions. A wide range of strategies have been proposed to address this, spanning from classical statistical techniques to advanced learning-based methods. Notably, CIR-driven approaches have shown strong potential for both NLoS identification and ranging error mitigation, particularly when integrated into data-driven models. However, many existing studies evaluate these methods in static or low-dynamic scenarios, or even in isolation from positioning systems, thereby neglecting their practical impact on localization performance. Moreover, the field continues to be constrained by the reliance on a narrow set of hardware platforms.

Building upon the strengths of CIR-driven error mitigation, the present study incorporates these techniques into a comprehensive UWB-based positioning framework implemented on a novel hardware platform. This integration enables a more realistic evaluation of their effectiveness under dynamic conditions and supports the development of more accurate and robust indoor localization systems.
