\begin{abstract}
\addchaptertocentry{\abstractname}
The increasing demand for precise indoor localization in areas such as robotics, healthcare, and asset tracking necessitates the development of high-accuracy indoor positioning systems (IPSs) due to the limitations of satellite-based technologies in indoor environments. Ultra-wideband (UWB) has emerged as a leading technology for IPS, offering centimeter-level ranging accuracy. However, UWB performance significantly deteriorates under Non-Line-of-Sight (NLoS) conditions due to multipath propagation and signal obstruction. In this work, we present the development of an end-to-end UWB-based IPS, implemented using the IEEE 802.15.4z-compliant hardware platform and evaluated on the task of trajectory reconstruction. To address NLoS-induced errors, we introduce A-REMNet, a Channel Impulse Response (CIR)-based deep neural network for data-driven range correction. The model is trained on a large, custom-collected dataset, which is made publicly available to support future research. Corrected range measurements are subsequently fused in an Extended Kalman Filter (EKF) framework to produce position estimates. With the proposed model integrated into the localization pipeline, the system is found to achieve median localization errors of \SI{3.4}{\centi\metre} in unobstructed conditions, \SI{4.5}{\centi\metre} in previously seen NLoS settings, and \SI{5.4}{\centi\metre} under NLoS conditions in an unseen environment, corresponding to relative improvements of \SI{36}{\percent}, \SI{59}{\percent}, and \SI{50}{\percent}, respectively, compared to the baseline performance without error mitigation.
\end{abstract}