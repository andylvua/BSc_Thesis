\chapter{Conclusion}

\section{Discussion}
The primary aim of this work was to design, develop, and evaluate a robust UWB-based indoor positioning system capable of maintaining high accuracy in challenging environments. Therefore, to address the inherent limitations of UWB under obstructed conditions, we placed a particular emphasis on improving ranging accuracy through CIR-driven NLoS error mitigation. To this end, we developed a fully integrated, end-to-end localization framework comprising three principal contributions.

First, a custom UWB-based ranging system was designed and implemented on a novel hardware platform built around the SynchronicIT SFM10 module, which incorporates the NXP Trimension OL23D0 UWB chipset. Unlike prior studies relying predominantly on Qorvo (formerly Decawave) transceivers, this work pioneers the use of a new UWB platform in a research context. In controlled experiments under LoS conditions, the system demonstrated high baseline accuracy, with median ranging errors kept under \SI{\pm 5}{\centi\metre}. The proposed TDMA-based synchronization across multiple anchors achieved a per-anchor measurement rate of \SI{67}{\hertz}.

Second, a comprehensive dataset for NLoS error mitigation and channel-state identification was collected under controlled indoor conditions. Data were gathered in both clear-LoS and dynamically induced human-occlusion NLoS scenarios, with an optical tracking system automatically labeling each measurement. The resulting dataset, comprising nearly one million measurements with synchronized CIRs, was published to support further research~\cite{yaroshevych_2025_rangecir}.

Third, a learning-based error mitigation module was designed and developed to predict measurement errors based on CIR data. Extending the REMNet architecture, proposed by Simone Angarano et al.~\cite{Simone2021UWB}, the proposed A-REMNet model incorporates complex-valued CIR inputs and a temporal self-attention mechanism. The predicted range biases were used to correct raw estimates, which were subsequently fused via an EKF to produce position estimates. The performance of the proposed pipeline was assessed through experimental trajectory reconstruction trials, achieving up to \SI{70}{\percent} median reduction in mean absolute localization error under NLoS conditions -- from \SI{15.9}{\centi\metre} to \SI{4.4}{\centi\metre} in the training environment, and up to \SI{58}{\percent} improvement -- from \SI{9.3}{\centi\metre} to \SI{3.8}{\centi\metre} in the unseen one, compared with raw, non-mitigated trajectories. Notably, even in LoS conditions, the error was still reduced by up to \SI{39}{\percent}, despite relatively low initial values -- from 5.0 to 3.0 cm.

These results further support the conclusion, well-established in prior work, that CIR-based, learning-driven mitigation can effectively reduce UWB ranging errors in realistic NLoS conditions. Notably, compared to existing literature, which often assesses error mitigation methods in isolation, we present a more integrated and empirically validated approach, evaluating the proposed solution within a complete positioning system prototype, operating under dynamic conditions.

\section{Limitations and future work}

Despite the promising results, this study has several limitations. The evaluation was constrained to two indoor environments, with NLoS conditions induced only by human occlusion. While representative of many practical scenarios, these occlusions tend to introduce relatively low ranging bias. Consequently, further studies are required to assess performance in various environments and with stronger obstructions, such as structural barriers or vehicles, commonly found in industrial and automotive contexts. Moreover, only a single anchor deployment configuration was explored. Systematic studies of anchor density, geometry, and mounting strategies, such as ceiling-mounted configurations, which may exacerbate antenna polarization mismatches or increase GDOP, are needed to characterize their impact on positioning accuracy.

To improve generality, future work should also aim to expand the empirical dataset to include a wider spectrum of deployment scenarios, environments, and obstruction types. Additionally, examining the feasibility of on-device deployment of the proposed error mitigation model, particularly on embedded systems with constrained computational resources, would be beneficial for real-world scalability. The combination of CIR-driven models with unsupervised or semi-supervised learning frameworks also presents a compelling direction for reducing reliance on labeled datasets. Lastly, integrating this approach with other sensing modalities, such as inertial or visual systems, may further improve the robustness of the proposed system.

