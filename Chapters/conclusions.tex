\chapter{Conclusions}

\section{Discussion}
The primary aim of this work was to investigate and develop a robust Ultra-Wideband (UWB)-based indoor localization system, with a particular emphasis on improving ranging accuracy through Channel Impulse Response (CIR)-driven Non-Line-of-Sight (NLoS) error mitigation. To address the inherent limitations of UWB systems under obstructed conditions, a CIR-based machine learning pipeline was proposed, implemented, and evaluated within a fully integrated localization framework.

To achieve this goal, several key tasks were defined and addressed. First, a custom UWB-based ranging system was designed and implemented on a novel hardware platform based on the SFM10 module with NXP Trimension OL23D0 chip. Unlike prior studies relying predominantly on Decawave (now Qorvo) transceivers, this work pioneers the use of a new UWB platform in a research context. The system demonstrated stable baseline accuracy, with median ranging errors under $\pm 5$ cm across distances up to 10 meters in line-of-sight (LoS) conditions, and standard deviation of approximately 3 cm. Time Division Multiple Access (TDMA) scheduling was employed for synchronized multi-anchor operation, achieving a combined measurement frequency of 270 Hz.

Second, a comprehensive dataset was collected under controlled indoor conditions. The data acquisition campaign involved both LoS and dynamically induced human-occlusion NLoS scenarios, and employed an optical tracking system for automated LoS/NLoS labeling. The resulting dataset, comprising nearly one million measurements with synchronized CIRs, was published to support further research in NLoS error mitigation and channel-state identification~\cite{yaroshevych_2025_rangecir}.

Third, a learning-based error mitigation module was developed. Extending the REMNet architecture~\cite{Simone2021UWB}, the proposed Attention-enhanced REMNet (A-REMNet) model incorporated complex-valued CIR inputs and multi-head self-attention mechanism. The model was trained to predict ranging errors, which were subsequently used to correct raw ToF-based distance estimates. The corrected measurements were fused over time via an Extended Kalman Filter (EKF), yielding the final trajectory estimate. Experimental evaluation demonstrated that the proposed model significantly improved ranging precision, achieving up to 58\% improvement in unseen environment, and up to 70\% in training one.

These findings confirm the effectiveness of CIR-driven learning-based techniques for mitigating UWB ranging errors under realistic NLoS conditions. Compared to existing literature, the proposed system advances the field by combining a real-time ranging architecture with error correction model, evaluated within a complete IPS prototype. Unlike prior studies which often assess error mitigation methods in isolation without integrating them into operational localization systems -- this work presents a more integrated and empirically validated approach.

\section{Limitations and future work}

Despite the promising results, the study has several limitations. The evaluation was constrained to a single indoor environment, with NLoS conditions induced only by human occlusion. While representative of many practical scenarios, these occlusions tend to introduce relatively low ranging bias. Consequently, further studies are required to assess performance in environments with stronger obstructions, such as structural barriers or vehicles, commonly found in industrial and automotive contexts. Moreover, only a single anchor deployment configuration was explored. The influence of anchor density, geometry, and mounting strategies -- such as ceiling-mounted configurations, which may exacerbate antenna polarization mismatches -- remains to be systematically evaluated, as these factors can significantly affect the Geometric Dilution of Precision (GDOP) and, consequently, the accuracy of position estimates.

Future work should also aim to expand the empirical dataset to include a broader range of deployment scenarios and obstruction types. Additionally, investigating the potential for on-device deployment of the A-REMNet model, particularly on embedded systems with constrained computational resources, would be beneficial for real-world scalability. The combination of CIR-driven models with unsupervised or semi-supervised learning frameworks also presents a compelling direction for reducing reliance on labeled datasets. Lastly, the integration of this approach with other sensing modalities, such as inertial or visual systems, may further improve robustness of the proposed system.
