\chapter{Introduction}

In recent decades, with the development of mobile communication and Internet-of-Things (IoT) technologies, the demand for indoor positioning systems (IPSs) has been steadily increasing, as they are crucial for a wide range of applications, including asset tracking, smart infrastructure, robotics, healthcare, and emergency response~\cite{indoor2022Farahsari}. IPSs address the challenge of accurately determining the position of objects or individuals in indoor environments, where traditional technologies, such as Global Positioning System (GPS) and other satellite-based systems, fail due to signal degradation caused by structural obstructions~\cite{alarifi2016ultra}.

Among the various approaches proposed for indoor localization, radio-frequency (RF) technologies, particularly Wi-Fi~\cite{Abbas2019Wifi}, Bluetooth Low Energy (BLE)~\cite{Jianyong2014BLE_RSSI}, and Ultra-wideband (UWB)~\cite{Simone2021UWB}, have gained the most widespread adoption due to compatibility with consumer devices and low deployment costs~\cite{leitch2023indoor}. Wi-Fi and BLE-based systems typically rely on Received Signal Strength Indicator (RSSI)~\cite{Palumbo2015RSS} or RSS-fingerprinting~\cite{Pelant2017RSSFinger} methods to estimate the location of the target, though BLE can also employ other methods, such as phase-based ranging~\cite{Dyhdalovych2025BLE}. RSS-based approaches estimate distance using the attenuation of signal strength, while fingerprinting involves pre-mapping signal strength patterns in the environment and comparing real-time measurements to this reference during localization.

Despite their popularity, narrowband technologies, including Wi-Fi and BLE, often suffer from limited accuracy, as they are highly susceptible to environmental interference and multipath effects, making them unsuitable for applications requiring high precision~\cite{Yang2021LOC, Zand2019PhaseBLE, Ahmed2024BLE}. Therefore, recent advancements have shifted attention towards UWB technology, which has emerged as a state-of-the-art solution for wireless indoor localization~\cite{Simone2021UWB}. Unlike other radio technologies, UWB operates over a wide frequency range and transmits extremely short pulses, minimizing interference and enabling high temporal resolution~\cite{Che2022UWB}. This capability allows it to distinguish between direct and reflected signals, thereby mitigating multipath propagation effects. By precisely identifying the first signal arrival, unlike RSS-based approaches, UWB systems can perform accurate time-of-flight measurements to estimate distances between transceivers~\cite{Qu_2023}. As a result, UWB is capable of achieving decimeter to centimeter-level accuracy, often outperforming its narrowband counterparts~\cite{Wang2023NLoS, Khan2022UWB}.

However, while UWB technology offers substantial improvements in localization accuracy, it still encounters significant challenges in complex indoor environments~\cite{Yang2024}. Although it performs well in mild multipath conditions, its effectiveness diminishes under severe Non-Line-of-Sight (NLoS) scenarios, where walls, furniture, or human bodies obstruct the direct path between transmitter and receiver~\cite{Wang2023NLoS}. In such cases, the direct signal component may be heavily attenuated or entirely blocked, making it difficult or impossible to detect reliably~\cite{Tran2022UWB, pei2024fcn}. As a result, the system may incorrectly identify a reflected signal as the first arrival, leading to considerable ranging errors and degraded localization performance.
