\chapter{Introduction}

In recent decades, with the development of mobile communication and Internet-of-Things (IoT) technologies, the demand for indoor positioning systems (IPSs) has been steadily increasing, as they are crucial for a wide range of applications, including asset tracking, robotics, healthcare, and emergency response~\cite{indoor2022Farahsari}. IPSs address the challenge of accurately determining the position of objects or individuals in indoor environments, where traditional technologies, such as global positioning system (GPS) and other satellite-based systems, fail due to signal degradation caused by structural obstructions~\cite{alarifi2016ultra}.

Traditional approaches for indoor localization are primarily based on WiFi~\cite{Abbas2019Wifi} or Bluetooth Low Energy (BLE)~\cite{Jianyong2014BLE_RSSI}, due to their cost-effectiveness and widespread presence in smartphones. WiFi-based localization typically relies on Received Signal Strength Indicator (RSSI)~\cite{Palumbo2015RSS} or fingerprinting~\cite{Pelant2017RSSFinger} methods. RSSI-based approaches estimate distance using the attenuation of signal strength over distance, while fingerprinting involves pre-mapping signal strength patterns in the environment and matching them during localization. BLE technology typically leverages RSSI and fingerprinting, similar to WiFi, however, it can also employ other methods, such as phase-based ranging~\cite{Zand2019PhaseBLE}. Phase-based solutions estimate the distance between transmitted and received signals using phase shifts, or use phase differences across multiple antennas to determine the signal's incoming angle~\cite{Dyhdalovych2025BLE}.

However, these technologies suffer from low accuracy, as they are highly susceptible to environmental interference and multipath effects, making them unsuitable for applications requiring high precision~\cite{Yang2021LOC, Zand2019PhaseBLE, Ahmed2024BLE}. Therefore, recent advancements have shifted attention towards Ultra-Wideband (UWB) technology, which has emerged as a state-of-the-art solution for wireless localization~\cite{Simone2021UWB}. UWB systems rely on precise Time-of-Flight (ToF) measurements to obtain the distance estimate between transceivers~\cite{Qu_2023}. Unlike other radio-frequency technologies, UWB operates over a wide frequency range and transmits extremely short pulses, minimizing interference and enabling high temporal resolution~\cite{Che2022UWB}. This allows UWB to distinguish between direct and reflected signals, achieving decimeter to centimeter-level accuracy, and surpassing the capabilities of narrowband counterparts such as WiFi and BLE~\cite{Wang2023NLoS, Khan2022UWB}. 

While UWB offers substantial improvements in localization accuracy, it still encounters significant challenges in complex indoor environments. Although it performs well in mild multipath conditions, its effectiveness diminishes under severe Non-Line-of-Sight (NLoS) scenarios, where walls, furniture, or human bodies obstruct the direct path between transmitter and receiver. In such cases, the direct signal component may be heavily attenuated or entirely blocked, making it difficult or impossible to detect reliably~\cite{Tran2022UWB, pei2024fcn}. As a result, the system may incorrectly identify a reflected path as the first arrival, leading to considerable ranging errors and degraded localization performance.
